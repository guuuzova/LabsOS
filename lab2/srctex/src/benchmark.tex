\section{Результаты}

Программа получает на вход три параметра: максимальное число потоков, доступный объём памяти и имя файла с 128-символьными шестнадцатеричными числами. На выходе она выводит одну строку — 128-символьное шестнадцатеричное представление округлённого среднего арифметического всех загруженных чисел, а также время выполнения в миллисекундах.
Промежуточные вычисления выполняются с точностью до 512 бит без потерь. Если при создании потоков или чтении чисел из файлов возникают ошибки, программа завершается сообщением в \texttt{stderr}.