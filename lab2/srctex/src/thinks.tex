\section{Выводы}

В ходе выполнения лабораторной работы были приобретены практические навыки в организации многопоточной обработки данных в операционных системах, распределения вычислительной нагрузки между потоками и работы с большими объёмами численных данных без разделяемого состояния.\
Была составлена и отлажена программа на языке C++, реализующая параллельное вычисление среднего арифметического для массива 512-битных целых чисел, представленных в шестнадцатеричном виде. Программа использует средства многопоточности (\texttt{pthread\_create, pthread\_join}), что обеспечивает корректную работу в Unix-подобных операционных системах и поддерживает кросплатформенность.\
В результате работы программа ограничивает объём загружаемых данных в соответствии с заданным лимитом памяти, распределяет числа между потоками, а каждый поток независимо вычисляет свою частичную сумму. Обмен данными между потоками полностью исключён во время расчётов. Финальное объединение результатов выполняется последовательно в главном потоке.\
Были обработаны возможные ошибки: некорректные аргументы командной строки, невалидные символы в числах, а также сбои при создании потоков. Экспериментально подтверждена эффективность параллельных вычислений: при увеличении числа потоков (до числа ядер) время выполнения сокращается почти линейно, особенно при обработке больших наборов данных.
\pagebreak