\section{Метод решения}

Программа загружает массив больших целых чисел (512-битных) из файла, распределяет их равномерно между заданным числом потоков и вычисляет сумму параллельно. Каждый поток накапливает локальную частичную сумму. После завершения всех потоков главный поток собирает локальные результаты, вычисляет среднее арифметическое и применяет арифметическое округление.Каждый поток пишет только в свою собственную структуру \texttt{ThreadData}, финальное суммирование происходит последовательно в главном потоке.

\section{Описание программы}

\texttt{main.cpp} --- точка входа в программу. Отвечает за парсинг аргументов командной строки. Создает и запускает рабочие потоки, распределяя между ними нагрузку. После завершения всех потоков вычисляет и округляет среднее,выводит результат и замер времени.

\texttt{pthread.h} --- объявление общего кроссплатформенного интерфейса для работы с потоками.

\vspace{1\baselineskip}
Основные функции:
\begin{itemize}
    \item \texttt{Thread::Thread(ThreadFunc func)} — конструктор, создаёт объект \texttt{ThreadInfo}
    \item \texttt{void Thread::Run(void* data)} — запускает поток.Обёртка над \texttt{pthread\_create()}.
    \item \texttt{void Thread::Join()} — ожидает завершения потока.\texttt{pthread\_join()}.
    \item \texttt{Thread::Thread(Thread\&\& other) noexcept} и \texttt{Thread\& operator=(Thread\&\&)} — передают владение потоком от одного объекта к другому, не запуская поток заново.
    \item \texttt{void Thread::Swap(Thread\& other)} — обменивает содержимое двух объектов \texttt{Thread}.
\end{itemize}

\vspace{2\baselineskip}

Главный поток (main) подготавливает вектор ThreadData, содержащий указатель на общий массив чисел и границы поддиапазона \texttt{[start, end)} для каждого потока. Затем он создаёт K=min(threadsCount,N) рабочих потоков. Каждый получает указатель на свою структуру \texttt{ThreadData} и выполняет функцию \texttt{thread\_func}.

В \texttt{thread\_func} каждый поток:
\begin{itemize}
    \item последовательно складывает все числа из своего диапазона в локальный \texttt{BigInt sum} с использованием метода \texttt{Add},
    \item не синхронизируется с другими потоками во время вычислений,
    \item завершает работу, оставляя результат в своей \texttt{ThreadData.sum}.
\end{itemize}

После вызова \texttt{Join()} для всех потоков главный поток последовательно суммирует все \texttt{threadData[i].sum} в \texttt{total\_sum}, затем вычисляет среднее с округлением — объединение результатов происходит в одном потоке, это исключает гонки и делает мьютексы избыточными.