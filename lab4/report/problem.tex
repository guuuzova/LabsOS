\section{Условие}

{\bfseries Цель работы:} \\
Приобретение практических навыков в:
\begin{itemize}
\item Создание динамических библиотек
\item Создание программ, которые используют функции динамических библиотек
\end{itemize}

{\bfseries Задание:} \\
Требуется создать динамические библиотеки, которые реализуют заданный вариантом функционал. Далее использовать данные библиотеки 2-мя способами:
\begin{enumerate}
\item Во время компиляции (на этапе «линковки»/linking)
\item Во время исполнения программы. Библиотеки загружаются в память с помощью интерфейса ОС для работы с динамическими библиотеками
\end{enumerate}

В конечном итоге, в лабораторной работе необходимо получить следующие части:
\begin{itemize}
\item Динамические библиотеки, реализующие контракты, которые заданы вариантом
\item Тестовая программа (программа №1), которая используют одну из библиотек, используя информацию полученные на этапе компиляции
\item Тестовая программа (программа №2), которая загружает библиотеки, используя только их относительные пути и контракты.
\end{itemize}

Провести анализ двух типов использования библиотек.

Пользовательский ввод для обоих программ должен быть организован следующим образом:
\begin{enumerate}
\item Если пользователь вводит команду «0», то программа переключает одну реализацию контрактов на другую (необходимо только для программы №2)
\item «1 arg1 arg2 … argN», где после «1» идут аргументы для первой функции, предусмотренной контрактами. После ввода команды происходит вызов первой функции, и на экране появляется результат её выполнения
\item «2 arg1 arg2 … argM», где после «2» идут аргументы для второй функции, предусмотренной контрактами. После ввода команды происходит вызов второй функции, и на экране появляется результат её выполнения
\end{enumerate}


{\bfseries Вариант:} 11 \\
Функции:
\begin{enumerate}
\item §	Рассчет производной функции cos(x) в точке A с приращением deltaX
    \begin{itemize}
    \item Float Derative(float A, float deltaX)
    \item Реализация 1: f'(x) = (f(A + deltaX) – f(A))/deltaX
    \item Реализация 2: f'(x) = (f(A + deltaX) – f(A-deltaX))/(2*deltaX
    \end{itemize}
\item Рассчет значения числа Пи при заданной длине ряда (K)
    \begin{itemize}
    \item float Pi(int K)
    \item Реализация 1: Ряд Лейбница
    \item Реализация 2: Формула Валлиса
    \end{itemize}
\end{enumerate}

