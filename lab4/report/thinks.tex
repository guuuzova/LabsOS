\section{Выводы}

В ходе выполнения лабораторной работы были получены практические навыки работы с динамическими библиотеками на языке C++. Были разработаны два контракта (\texttt{proizv\_contract.h}, \texttt{pi\_contract.h}) и четыре реализации в виде динамических библиотек , которые используют разные алгоритмы при едином интерфейсе.

Программа №1 продемонстрировала использование динамических библиотек с привязкой на этапе линковки, что упрощает запуск, но требует пересборки при смене реализации. Программа №2 реализует явную загрузку библиотек во время выполнения, что позволяет переключать реализацию алгоритмов без перекомпиляции.

Был реализован абстрактный слой \texttt{dynamic\_library.cpp}, облегчающий возможную адаптацию программы для других ОС.

Программа демонстрирует:
\begin{itemize}
    \item использование динамических библиотек при линковке (program1);
    \item динамическую загрузку и смену реализаций во время выполнения (program2);
    \item абстракцию системных вызовов для обеспечения кроссплатформенности;
    \item обработку ошибок;
    \item корректное завершение работы при возникновении ошибок.
\end{itemize}

В лабораторной работе были реализованы два подхода к использованию динамических библиотек.

\textbf{Подход 1: линковка на этапе компиляции (programram1)}
\begin{itemize}
\item Привязка к библиотекам выполняется линковщиком во время сборки.
\item Меньше накладных расходов при запуске.
\item Недостаток: для смены реализации требуется менять настройки сборки и пересобирать программу.
\end{itemize}

\textbf{Подход 2: явная загрузка во время выполнения (program2)}
\begin{itemize}
\item Библиотеки загружаются в момент работы программы.
\item Можно переключать реализацию контрактов по команде пользователя без перекомпиляции.
\item Небольшие дополнительные накладные расходы на вызовы функций работы с динамическими библиотеками, но подход более гибкий и маштабируемый.
\item Необходимо вручную обрабатывать ошибки.
\end{itemize}