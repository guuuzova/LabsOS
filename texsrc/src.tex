\section{Метод решения}
Parent запрашивает у пользователя имя файла,создает pipe,передает имя полученного файла child(ребенку).Child в stdin читает числа из файла,складывает их и stdout направляет в pipe возвращает результат. Parent выводит ответ в stdout.

Кроссплатформеннность достигается разной реализацией функций,описанных в файле os.h, в файлах os\_unix.cpp, os\_win.cpp

\section{Описание программы}
Разделение по файлам, описание основных типов данных и функций. Обязательно написать используемые системные вызовы.

\texttt{parent.cpp} --- точка входа в программу. Создаёт дочерний процесс, передает ему файл с цифрами и через канал выводит итоговый результат сложения чисел.

\texttt{child.cpp} --- исполняемый файл дочернего процесса. Читает из файла цифры,складывает их и передает результат в канал.

\texttt{os.h} --- объявление функций-обёрток над системными вызовами ОС  для управления процессами и каналами. 

\texttt{os\_unix.cpp} --- реализация для Unix-подобных ОС.

\texttt{os\_win.cpp} --- реализация для Windows.

Основные функции:
\begin{itemize}
\item \texttt{intCreateChildProcess(const StartProcess\& args);} --- создаёт дочерний процесс.Заменяет \texttt{fork()} + \texttt{dup2()} + \texttt{close()} + \texttt{execl()}
\item \texttt{bool CreatePipe(PipeHandle\& readpipe, PipeHandle\& writepipe);} --- создаёт канал для передачи данных.Возвращает дискрипторы для записи и чтения из канала.Заменяет \texttt{pipe()}.
\item \texttt{int ReadPipe(PipeHandle pipe, void* buf, int count)} --- читает из канала определенное количество байт в буфер.Заменяет \texttt{read()}
\item \texttt{void ClosePipe(PipeHandle pipe);} --- закрывает конец канала.Заменяет \texttt{close()}
\item \texttt{Exit(int code);} --- завершает текущий процесс. Обёртка над \texttt{\_exit()}.
\end{itemize}

Архитектура взаимодействия:

Родительский процесс (\texttt{parent.cpp}) считывает строку от пользователя с названием файла, передает его содержимое в stdin дочернего процесса (\texttt{child1}),который выполняет сложение чисел и через канал \texttt{pipe} возвращает родительскому процессу результат сложения.Родитель читает результат из \texttt{pipe} и выводит его на экран.
