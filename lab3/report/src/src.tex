\section{Метод решения}
Parent.cpp запрашивает у пользователя имя файла, создаёт разделяемую память и child. Ребенок читает числа из файла, складывает их и записывает результат в виде строки в разделяемую память. Потом он посылает родителю сигнал SIGUSR2, говоря о готовности ответа. Родительский процесс ожидает этот сигнал, после чего извлекает и выводит результат.

\section{Описание программы}
Разделение по файлам, описание основных типов данных и функций. Обязательно написать используемые системные вызовы.

\texttt{parent.cpp} --- точка входа в программу.Запрашивает имя файла, создаёт разделяемую память, запускает дочерний процесс, ожидает сигнала и выводит результат.

\texttt{child.cpp} --- исполняемый файл дочернего процесса.Получает имя файла и имя разделяемой памяти, читает числа из файла, суммирует их и записывает результат в разделяемую память, посылает сигнал родителю.

\texttt{os.h} --- объявление функций-обёрток над системными вызовами ОС  для управления процессами, сигналами и разделяемой памятью. 

\texttt{os.cpp} --- реализация для Unix-подобных ОС.

\texttt{exception.cpp} --- объявление пользовательчких исключений.

Основные функции:
\begin{itemize}
\item \texttt{CreateSharedMemory(name, size)} --- создаёт объект разделяемой памяти с заданным именем и размером (вызывает \texttt{shm\_open}, \texttt{ftruncate}, \texttt{mmap}).
\item \texttt{Open(name, size)} — подключается к существующей разделяемой памяти (выполняет \texttt{shm\_open} + \texttt{mmap}).
\item \texttt{Destroy(shm)} — деинициализирует и удаляет разделяемую память (\texttt{munmap}, \texttt{close}, \texttt{shm\_unlink}).
\item \texttt{Unmap(shm)} — отсоединяет текущий процесс от разделяемой памяти (\texttt{munmap} + \texttt{close}).
\item \texttt{CreateChildProcess(exe, arg1, arg2)} — создаёт дочерний процесс через \texttt{fork} + \texttt{execl}.
\item \texttt{SetSignalHandler(signum, handler)} — регистрирует обработчик сигнала (\texttt{signal}).
\item \texttt{SendSignal(pid, signum)} — посылает сигнал указанному процессу (\texttt{kill}).
\item \texttt{WaitSignal()} — ожидает поступления сигнала через цикл с \texttt{pause()} и флагом \texttt{signal\_flag}, устанавливаемым в обработчике.
\item \texttt{WaitChild(pid)} — ждёт завершения дочернего процесса (\texttt{waitpid}).
\end{itemize}

Архитектура взаимодействия:

Родитель создаёт разделяемую память /lab размером 1024 байта.Устанавливает обработчик для сигнала SIGUSR2 (символ CONFIRM).Запускает дочерний процесс, передавая ему имя файла и имя разделяемой памяти.Блокируется в WaitSignal(), ожидая сигнала от ребёнка.Дочерний процесс подключается к той же разделяемой памяти. Вычисляет сумму чисел из файла.Преобразует результат в строку и копирует в разделяемую память.Посылает SIGUSR2 родителю через SendSignal(GetParentPID(), CONFIRM).Отключается от разделяемой памяти и завершается.Родитель читает строку из shm.ptr и выводит её.После завершения ребёнка (WaitChild) и чтения результата — освобождает разделяемую память.